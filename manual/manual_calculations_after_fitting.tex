\chapter{Calculations after fitting}
\label{ch:IllumCorrectionCalcsAfterFitting}
The fitting results are first used to correct the data on catalog basis via
\begin{eqnarray}
m_{app,corr} = m_{app} - \epsilon
\end{eqnarray}
with $\epsilon$ defined as in equation \ref{eqn:residual}. Also the residuals after fitting, including errors (via Gaussian error propagation), are calculated accordingly.\\
Additionally, the center position of the illumination correction can be obtained by the derivative of equation \ref{eqn:residual} via
\begin{eqnarray}
\frac{\partial \epsilon}{\partial x}  \stackrel{!}= 0 = 2Ax + Cy + D\label{eqn:delx} \\
\frac{\partial \epsilon}{\partial y}  \stackrel{!}= 0 = 2By + Cx + E\label{eqn:dely}
\end{eqnarray}
Solving equation \ref{eqn:delx} for y and \ref{eqn:dely} for x:
\begin{eqnarray}
x = -\frac{2By + E}{C}\label{eqn:x}\\
y = -\frac{2Ax + D}{C}\label{eqn:y}
\end{eqnarray}
Inserting result \ref{eqn:y} into equation \ref{eqn:x} and this afterwards again into \ref{eqn:y} yields:
\begin{eqnarray}
x_{max} & = & \frac{CE-2BD}{4AB-C^{2}}\label{eqn:ResultX}\\
y_{max} & = & \frac{2AE-CD}{C^{2}-4AB}\label{eqn:ResultY} .
\end{eqnarray}
Afterwards, several statistics of the residuals before and after fitting are calculated for each individual chip and for all chips combined:
\begin{itemize}
\item mean according to equation \ref{eqn:mean}
\item minimum
\item maximum
\item number of data points
\item variance according to equation \ref{eqn:variance}
\item standard deviation according to equation \ref{eqn:sigma}
\item number of objects that are compatible with 0 within a $1\sigma$-error
\item percent of objects that are compatible with 0 within a $1\sigma$-error (from 0 to 1)
\item center position of the illumination correction (maximum of the polynomial) according to equation \ref{eqn:ResultX} and \ref{eqn:ResultY} including errors.
\end{itemize}
The following values are calculated only for the entire camera for the ellipse:
\begin{itemize}
\item length of the minor axis
\item length of the major axis
\item ellipticity
\item numerical ellipticity
\item Rotation w.r.t. the y-axis, counted clock-wise (CW)
\item number of pixels that have a correction value between (at the moment) -0.015 and 0.0mag. These values are chosen such that in all illumination corrections, at least for KiDS observations, a certain number of pixels fulfil these requirements.
\item last value, given in percent (from 0 to 1).
\end{itemize}
